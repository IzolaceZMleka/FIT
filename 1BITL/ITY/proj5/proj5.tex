\documentclass[fyma2,pdf,final]{prosper}

\usepackage[czech]{babel}
\usepackage[utf8]{inputenc}
\usepackage[T1]{fontenc}
\usepackage{picture}
\usepackage{graphics}

\newcommand{\uv}[1]{\quotedblbase #1\textquotedblleft}

\slideCaption{\textit{Terorismus, Jan Pavlica}}
\DefaultTransition{Wipe}

\begin{document}

\title{Terorismus}
\author{Jan Pavlica}
\email{xpavli78@stud.fit.vutbr.cz}
\institution{Vysoké učení technické v~Brně\\Fakulta informačních technologií}
\date{10. května 2015}
\maketitle

\begin{slide}{Vymezení pojmu}
\begin{description}
	\item[Terorismus] \hfill \\
		\uv{Systematické užívání teroru, zvláště ve významu nátlaku}
	\item[Teror] \hfill \\
		\uv{Násilné či destruktivní chování jako bombardování skupinami ve snaze zastrašit obyvatelstvo nebo vládu, aby schválila jejich požadavky}
\end{description}
\begin{center}
\includegraphics[scale=2]{target17.eps}
\end{center}
\end{slide}

\begin{slide}{Aspekty terorismu}
\begin{enumerate}
	\item Akce jsou promyšlené a~mají za úkol zastrašit
	\item Teror je zaměřen na více lidí
	\item Náhodné symbolické cíle a~civilní obyvatelstvo
	\item Je chápán jako vybočení z~normálu
	\item Cílem je ovlivnit politické chování
\end{enumerate}
\end{slide}

\begin{slide}{Cíle terorismu}
\begin{enumerate}
	\item Reklamní cíl
		\begin{itemize}
		\item Upoutání pozornosti
		\end{itemize}
	\item Jednorázový násilný akt
		\begin{itemize}
		\item Dosažení konkrétních cílů
		\end{itemize}
	\item Strategický cíl
		\begin{itemize}
		\item Destabilizace režimu
		\end{itemize}
\end{enumerate}
\end{slide}

\begin{slide}{Metody užívané teroristy}
\begin{itemize}
	\item Atentáty 
	\item Bombové útoky
	\item Únosy osob
	\item Biologické a chemické útoky
	\item Útok pomocí civilních dopravních prostředků
\end{itemize}
\end{slide}

\begin{slide}{Dělení z hlediska preferovaných cílů}
\begin{itemize}
	\item tzv. revoluční
	\item fašistický
	\item separatistický
	\item náboženský
\end{itemize}
\end{slide}

\begin{slide}{Regiony spojené s problémy s terorismem}
\begin{table}[h]
\begin{center}
\catcode`\-=12
\begin{tabular}{|c|c|c|} \hline
    \textbf{Asie} & \textbf{Afrika} & \textbf{Evropa}\\ \hline
    Palestina/Izrael & Alžírsko & Severní Irsko\\
    Kašmír & Zimbabwe & Baskicko\\
    Mindango & Angola & \\
    Jižní Thajsko &  & \\
    Irák &  & \\
    Afghánistán &  & \\
    Čečensko &  & \\ \hline
\end{tabular}
\end{center}
\end{table}
\end{slide}

\begin{slide}{Teroristické organizace}
\begin{itemize}
  \item\textbf{Hizballáh} - Libanon, proti Izraeli
  \item\textbf{Hamas} - Palestina, proti Izraeli
  \item\textbf{Fatah} - umírněnější než Hamas, chce dohodu s Izraelem
  \item\textbf{Al-Kajda} - teroristická organizace s celosvětovou působností
  \item\textbf{ETA} - Španělsko, odtržení Baskicka
  \item\textbf{IRA} - snaha sjednotit Irsko
  \item\textbf{Om-šinrikjo} - Japonsko snaha odstranit císařství
\end{itemize}
\end{slide}

\begin{slide}{}
\vspace{54pt}
\begin{center}
{\large Děkuji za pozornost}
\end{center}
\end{slide}

\end{document}
